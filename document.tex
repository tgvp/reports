\documentclass[a4paper,11pt]{report}
\usepackage[utf8]{inputenc}
\usepackage{setspace}
\usepackage[T1]{fontenc}
\usepackage{fix-cm}
\usepackage{url}
\usepackage[Glenn]{fncychap}
\usepackage[hmargin={2.5cm}]{geometry}
\renewcommand{\familydefault}{\sfdefault}
\usepackage{blindtext}
\usepackage{tabularx}
\usepackage{tabulary}
\usepackage{booktabs}
\usepackage{hyperref}
\usepackage[all]{hypcap}
\usepackage{enumitem}
\usepackage{xspace}

%test
\renewcommand\FmN[1]{}
%opening
\title{1st Technical Report}

\begin{document}

\maketitle
%Testing third commit by Priest
\chapter{Introduction and PhD Project Context}

The research work performed in this project is closely aligned with the ALEXANDRIA 
and iCrawl projects at the L3S Research Center -  
Leibniz Universit\"at Hannover\footnote{The L3S Research Center: \url{www.L3S.de}}.  
%
The L3S Research Center, the host institution of the PhD project, 
is one of the leading institutions
focusing on basic and applied research in the field of Web Science.  
% 

The ALEXANDRIA project \footnote{The ALEXANDRIA project: \url{http://alexandria-project.eu/}} 
aims at creating a foundation for temporal retrieval, exploration and analytics in Web archives.
%
Web archives are a rich source of data since they reflect rapid evolution
of the digital world. The interest in using the data
stored within Web archives has been observed by researchers from different
areas, such as history, sociology and law \cite{icrawlRequirements}. For example, 
the discussion in \cite{Brugger14} indicates that Web archives are an important source for
communication and media history, as well as within historiography in general.
%
In the context of Web archives, unique Web data collections available within 
the ALEXANDRIA project represent a great opportunity to create new
methods to explore and analyse the large scale data within Web archives.
It also provides contacts with high level
research scientists that has been crucial to the development of the PhD project.
%
While ALEXANDRIA focuses on already existing data collections within Web archives,
the iCrawl project\footnote{\url{http://icrawl.l3s.uni-hannover.de/}} 
aims at creation of such collections efficiently on demand, 
using Web and Social Web data. In particular, the role of Social Media to achieve
better temporal and topical coherence of the resulting collections is in the  
main research focus of the project.

In this reporting period, the student worked on the development of novel semantic analytics techniques 
to enable efficient light-weight access to large scale Web and Social Web data collections 
available in the context of the ALEXANDRIA and the iCrawl projects. 
%
In the following sections, we present a detailed description regarding the PhD work
developed at the L3S during this period. 
Then we present the work plan for the second year. 
%
Table~\ref{tab:work} illustrates the detailed information about the PhD
candidate at the L3S Research Center.

\begin{table} [!htb]
\begin{center}
\begin{tabular}{|l|l|l}
\cline{1-2}
\multicolumn{1}{|c|}{Name} & \multicolumn{1}{c|}{Thiago Gustavo Vieira de Paiva}
&
\\
\cline{1-2}
\multicolumn{1}{|c|}{Report number} & \multicolumn{1}{c|}{1} &  \\
\cline{1-2}
\multicolumn{1}{|c|}{Area} & \multicolumn{1}{c|}{Computer Science} &  \\ 
\cline{1-2}
\multicolumn{1}{|c|}{Institution} & \multicolumn{1}{c|}{Universidade Federal do
Maranhão - Brasil} & \\
\cline{1-2}
\multicolumn{1}{|c|}{Department} & \multicolumn{1}{c|}{Informática} &\\
\cline{1-2}
\multicolumn{1}{|c|}{Professional Address} & \multicolumn{1}{c|}{
Avenida dos Portugueses, 1966 - Bacanga, São Luís - MA, 65065-545, Brasil} &
\\
\cline{1-2}
\multicolumn{1}{|c|}{E-mail} &
\multicolumn{1}{c|}{thiago.gustavo@gmail.com} &\\
\cline{1-2}
\multicolumn{1}{|c|}{Research Project} &
\multicolumn{1}{c|}{ALEXANDRIA}
&
\\
\cline{1-2}
\multicolumn{1}{|c|}{Title} & \multicolumn{1}{|c|}{Master} \\
\cline{1-2}
\multicolumn{1}{|c|}{Supervisor} & \multicolumn{1}{|c|}{Prof. Dr. Wolfgang
Nejdl | nejdl@l3s.de} \\
\cline{1-2}
\multicolumn{1}{|c|}{Advisor/Mentor} & \multicolumn{1}{|c|}{Dr. Eirini Ntoutsi
| ntoutsi@kbs.uni-hannover.de} \\
\cline{1-2}
\multicolumn{1}{|c|}{Advisor/Mentor} & \multicolumn{1}{|c|}{Msc. Tarcísio Souza
Costa | souza@l3s.de}
\\
\cline{1-2}
\multicolumn{1}{|c|}{Period of Evaluation} &
 \multicolumn{1}{c|}{January, 2017$\rightarrow$March, 2017} \\
\cline{1-2}
\end{tabular}

\caption{Student description}
\end{center}
\label{tab:work}
\end{table}


%testing
\chapter{Description of Research Activities}

\section{Research and development activities}

In the first month, the student has been involved in some of the key projects
at L3S that focus on classification of News collections for instance, The New
York Times and UK News datasets.

\subsection{R\&D activities in the context of News Classification}
In the initial phase, it was studied the literature related to the 
problems of Text and News Multilabel Classification.
He also became familiar with softwares as well as 
with the Big Data research infrastructure of the L3S Research Center. 
Also in this period, the student started to learn new Big Data technologies, 
such as Scala and Spark.
%
\subsection{R\&D activities in the context of the ALEXANDRIA project}
The student became familiar with the work plan of the ALEXANDRIA
project, the literature involving Multilabel Classification and information
extraction tools in order to annotate the UK News dataset accordingly to the
ground truth generated by the New York Times News' classifier.
%
At the same period the student also learned a distributed indexing technology, named Elastic
Search. The index was
developed and made available in Kibana (a user-friendly Elastic Search graphical user
interface). 
%
The student developed categorization of 
URL search results by domain and mime type, performed index optimization such as setting some fields as analyzed 
by the search engine as well as refinement of search engine results to retrieve documents in certain 
periods of time. 
%
The index was further improved by inserting mime type, domain and
improvements related to temporal expressions automatically 
extracted from the URLs, also changes in the main
structure were performed. This index should be used as a starting point for the iCrawl
to find relevant documents within the archive. 

\textbf{Named entity extraction from URLs:}
Further semantic analytics of the URLs included named entity extraction from URLs. 
The paper describing this work was accepted for publication at the 
1st International Keystone Conference
\footnote{\url{http://www.keystone-cost.eu/ikc2015/about.php}} at the beginning of
July 2015. 
% lead to a publication 
% work on this dataset The student is the main author of the paper 
% % started writing a paper together with the other colleagues from
% % the iCrawl project, 
In this paper, the student analysed a subset of URLs, named as Popular German
Web. Such a subset is related to popular domains according to Alexa
ranking\footnote{\url{http://www.alexa.com}}. In this paper we measure the
quantity of URLs that are alive (the status code starts with ``2'') per year and per domain, 
also counting in which domains categories they are included, as Sports, Games,
Computes and so on. We also extracted temporal expressions
and entities using only the information within URLs.
%
We continue working on this topic to further improve the results, for instance by adding better
pre-processing and post-filtering procedures in the URLs before applying state
of the art techniques to extract entities, as Stanford Named Entity Recognition
(NER) \footnote{\url{http://nlp.stanford.edu/software/CRF-NER.shtml}}. Using
those procedures we reach a precision up to 85\% in our dataset. 
%
Further improvements such as evaluating
precision and filtering on a sample of distinct URLs, instead of captures and
presentation of a more detailed description of how the precision was measured are currently 
being performed by the student as a part of the camera-ready preparation
and to address the review comments.

\subsection{Artificial Bee Clustering Search (ABCS) algorithm}
In addition to the research activities and publications in the ALEXANDRIA project, during this reporting period 
the student 
received an acceptance message from an extended version of a paper to be published in
a journal, the Artificial Bee Clustering Search (ABCS) algorithm \cite{costaiwann13}. 
After the final version submitted, the journal paper was published \cite{costa2014}. 

\section{Integration in the L3S research environment}
At the beginning of the first month at the L3S Research Center, the student 
joined the team lead by Dr. Eirini Ntoutsi and Msc. Tarcísio Souza Costa who
were appointed as the student advisors by Prof. Wolfgang Nejdl. 

Regular meetings with the advisors, and with Prof. Nejdl
as well as regular participation in the ALEXANDRIA project meetings were part
of the student integration in the L3S research environment.

These meetings were very helpful for the student that could obtain feedback on
his work, in particular with respect to further improvement of the results and promising research directions for future work.

\section{Community services}

The student worked as a sub-reviewer under supervision of Dr. Thomas Risse for the 
Joint Conference on Digital Libraries 2015 (JCDL 2015). That was a good experience since
the paper student reviewed was related to the PhD work and was
very useful to improve the student knowledge in the field.

\section{Courses}

In winter semester, the student followed the course ``Foundations of Information Retrieval"
including lectures and exercises offered by Prof. Nejdl. This lecture provides a solid 
foundation to better understand Information Retrieval concepts and terminology
used in the context of the ALEXANDRIA project and also to become familiar with the  
typical evaluation methods. 

In summer semester, the student participated in the course 
``Advanced Methods of Information Retrieval'' conducted by Dr. Elena Demidova. In this course
Information Retrieval specific methods were presented from the practical point of view, 
including implementation exercises and challenging research questions during the lectures.
This course was very useful for the student to become familiar with a diversity
of relevant topics in the field as well as open research directions relevant to the doctoral
study.

At the moment there are no grades for these courses, that is the reason the
doctorate transcript will not be provided, only in next annual report. The
student will take the information retrieval exams still in current month and
on September.
\section{Accepted publications}

\textbf{Title:} Semantic URL Analytics to Support Efficient Annotation of Large Scale Web Archives. \\
\textbf{Authors:} Tarcísio Souza, Elena Demidova, Thomas Risse, Helge Holzmann, Gerhard Gossen and Julian Szymanski. \\
\textbf{To appear in:} Proceedings of the 1st International Keystone Conference, Coimbra, Portugal, September 2015.\\
\textbf{Abstract:}
``Long-term Web archives comprise Web documents gathered over longer time periods and can easily reach hundreds of terabytes in size.
Semantic annotations such as named entities can facilitate intelligent access to the Web archive data. 
However, the annotation of the entire archive content on this scale is often infeasible. The most efficient way to access the documents within Web archives is provided through their URLs, which are typically stored in 
dedicated index files. The URLs of the archived Web documents can contain
semantic information and can offer an efficient way to obtain initial semantic annotations for the archived documents. 
In this paper, we analyse the applicability of semantic analysis techniques such as named entity extraction
to the URLs in a Web archive. We evaluate the precision of the named entity
extraction from the URLs in the Popular German Web dataset and analyse the
proportion of the archived URLs from 1,444 popular domains in the time interval from 2000 to 2012 to which these techniques are applicable.
Our results demonstrate that named entity recognition can be successfully applied to a large number of URLs in 
our Web archive and provide a good starting point to efficiently annotate large scale collections of Web documents.''



\chapter {Conclusion and Outlook}

During the first reporting period, the student worked on the problems 
of light-weight semantic annotation of large web collections in the 
context of the ALEXANDRIA and the iCrawl projects. The student learned 
state-of-the-art Big Data processing technologies,
developed novel approaches for Web data analytics, and published the results.

The first period at the L3S Research Center was a really rich experience, since the student had
an opportunity to study different research areas, learn new technologies as
well as working with top-level researchers.
%
During this period the student could find a well defined working approach
implemented by the L3S, where regular meetings with the supervisor (one per month)
and the advisor (one per week) were performed, which was very useful to
improve the student knowledge, also helping him to improve the PhD work that has
been developed in last months. 
%
% %
% The difficulties the student faced in the first
% time were related to the organization of the work, 
% in particular with respect to estimating the time needed to become proficient with the 
% current Big Data technologies and software.
% %
% It was a satisfactory period and the first developed work
% was just a starting point for what can be done in next periods.

In the second year, the student will continue research in the context of the ALEXANDRIA 
and the iCrawl projects. In particular, he will further develop and evaluate semantic analysis 
techniques for URLs, such as terms, named entities and 
temporal expressions extraction and correlation. Furthermore, he will develop novel methods to identify
interconnections between such semantic information spread across URLs
to enable light-weight URL-based sub-collection extraction of information related to events
from large unfocused Web data collections.
In this context the student will rely on machine learning and 
statistical techniques as well as soft computing - based algorithms. 

\onehalfspace
\bibliography{bibliografia}{}
\bibliographystyle{plain}

\end{document}

